%!TEX program = xelatex
\documentclass[a4paper,12pt]{NPUResearchProposal}
\usepackage[top=2.2cm, bottom=3cm, left=2.4cm, right=2.4cm]{geometry}
\usepackage{indentfirst}
\usepackage{extpfeil}
%\usepackage{times}  % Times fonts
\usepackage{titlesec}
\usepackage{hyperref}
\usepackage{graphicx}
\usepackage{subfigure}
\usepackage[ruled,vlined,norelsize]{algorithm2e}
\usepackage{esint}
\usepackage{amssymb,amsmath}
\usepackage{amsbsy}
\usepackage{amsthm}
\usepackage{type1cm}
\usepackage{type1ec}
% \usepackage{stmaryrd}
\usepackage[square,sort&compress,comma,numbers]{natbib}

%\usepackage{javen}
\usepackage{bbm}

\newcommand{\NP}{\mathcal{NP}}
\newcommand{\linespacing}{1.5}

% \xeCJKsetup{CJKglue=\hspace{0pt plus .08 \baselineskip }}  
\titleformat{\section}{\Large\bfseries}{\thesection.}{1em}{}

\makeatletter
\renewenvironment{thebibliography}[1]
     {\section*{\refname}% <--- outcommented
      \@mkboth{\MakeUppercase\refname}{\MakeUppercase\refname}%
      \list{\@biblabel{\@arabic\c@enumiv}}%
           {\settowidth\labelwidth{\@biblabel{#1}}%
            \leftmargin\labelwidth
            \advance\leftmargin\labelsep
            \@openbib@code
            \usecounter{enumiv}%
            \let\p@enumiv\@empty
            \renewcommand\theenumiv{\@arabic\c@enumiv}}%
      \sloppy
      \clubpenalty4000
      \@clubpenalty \clubpenalty
      \widowpenalty4000%
      \sfcode`\.\@m}
     {\def\@noitemerr
       {\@latex@warning{Empty `thebibliography' environment}}%
      \endlist}
\makeatother

\begin{document}

\title{西北工业大学博士开题报告模板}

% Separate multiple authors with "\and".
% Use the \thanks command to give your address
\author{作者}
\stunumber{2010xxxxxx}
\College{计算机学院}
\Major{计算机科学与技术}
\Degree{博士}
\Supervisor{XXX$\quad$教授}
\Classification{统分}
\Date{201~年~~月~~日}
\isbn{XXX-XXX-XXX-X}    
%\FundReasearch %基础研究
\AppResearch %应用研究
%\EngResearch %工程技术
%\CroResearch %跨学科研究
\date{\today}       
\maketitle

\begin{spacing}{\linespacing}

\section{研究背景与意义}
\label{sec:background}
公式示例见\eqref{eq:eq_example}。
\begin{equation}
  \label{eq:eq_example}
  \begin{split}
    &\oiint{\scriptstyle\partial \Omega
    }\mathbf{E}\cdot\mathrm{d}\mathbf{S} = \frac{1}{\varepsilon_0}
    \iiint_\Omega \rho \,\mathrm{d}V\textit{,}\\
   & \oiint{\scriptstyle \partial \Omega
    }\mathbf{B}\cdot\mathrm{d}\mathbf{S} = 0\textit{,}\\
    &\oint_{\partial \Sigma} \mathbf{E} \cdot
    \mathrm{d}\boldsymbol{\ell} = - \frac{d}{dt} \iint_{\Sigma}
    \mathbf{B} \cdot \mathrm{d}\mathbf{S} \textit{,}\\
    &\oint_{\partial \Sigma} \mathbf{B} \cdot \mathrm{d}\boldsymbol{\ell} = \mu_0 \iint_{\Sigma} \mathbf{J} \cdot \mathrm{d}\mathbf{S} + \mu_0 \varepsilon_0 \frac{d}{dt} \iint_{\Sigma} \mathbf{E} \cdot \mathrm{d}\mathbf{S}\textit{。}
  \end{split}
\end{equation}
多个公式示例见
\begin{subequations}
  \begin{align}\label{eq:eq_example2}
    \nabla \cdot \mathbf{E} = &\frac {\rho} {\varepsilon_0}\textit{,}\\
    \nabla \cdot \mathbf{B} =& 0\textit{,}\\
    \nabla \times \mathbf{E} =& -\frac{\partial \mathbf{B}} {\partial
      t}\textit{,}\\
    \nabla \times \mathbf{B} = &\mu_0\left(\mathbf{J} + \varepsilon_0 \frac{\partial \mathbf{E}} {\partial t} \right) \textit{。}
  \end{align}
\end{subequations}
插图示例见图\ref{fig:fig_example}。
\begin{figure}[!h]
  \centering
  \includegraphics[width=0.35\textwidth]{example.eps}
  \caption{插图示例}
  \label{fig:fig_example}
\end{figure}

参考文献示例如下\citep{rudin1964principles}。
\section{研究现状}
\label{sec:current_state}
\section{研究内容}
\label{sec:contents_research}
\section{研究难点}
\label{sec:keypoints}
\section{研究方案}
\label{sec:roadmap}
\section{进度安排}
\section{预期的创新点与成果}
\end{spacing}

\bibliographystyle{IEEEtranN}   % (uses file "plain.bst")
\bibliography{ref}      % expects file "myrefs.bib"
\end{document}

%%% Local Variables: 
%%% coding: utf-8
%%% mode: latex
%%% TeX-engine: xetex
%%% End: 
